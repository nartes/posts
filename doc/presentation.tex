\documentclass[unicode]{beamer}

\mode<presentation>
{
  \usetheme{Warsaw}
}

\usepackage[utf8]{inputenc}
\usepackage[T2A]{fontenc}
\usepackage{amssymb, amsmath, hyperref}
\usepackage[main=russian,english]{babel}

\title{Аутентификация в сети по средством PEAP}
\author{Сергей Синяк}
\date{29.10.2016}

\begin{document}

\begin{frame}
  \titlepage
\end{frame}

\section{Задача}
\begin{frame}
  Имеется wpa-eap хотспот с клиентской аутентификацией
  по средством PEAP. Требуется устранить проблему подключения
  через wpa\_supplicant v2.5 на Arch Linux.
\end{frame}

\section{Этапы решения}
\begin{frame}
  1) В Апреле 2016 пытался собирать логи и обсуждать их с разработчиками
    wpa\_supplicant

    https://bbs.archlinux.org/viewtopic.php?id=211619 ---

    [SOLVED] eduroam issue (wpa\_supplicant rollback)

    https://bbs.archlinux.org/viewtopic.php?id=211842 ---

    wpa\_supplicant, eduroam nl80211: set\_key failed; err=-22 Inval

    http://lists.infradead.org/pipermail/hostap/2016-April/035548.html ---

    eduroam disconnects with reason=3 (WPA-EAP, PEAP, MSCHAPv2)
\end{frame}

\begin{frame}
  2) С конца авгутса 2016 решил реализовать эксплоит для PEAPv0,v1
    на базе публикации:

    "Man-in-the-Middle in Tunneled Authentication Protocols"
    от 2008 года ---

    https://eprint.iacr.org/2002/163.pdf
\end{frame}

\section{Источники информации}
\begin{frame}
  1) Исходный код hostap

  http://w1.fi/cgit/hostap/

  2) Книга про TLS и SSL

  "Bulletproof SSL and TLS"\ by Ivan Ristic, 2014
\end{frame}

\section{План}
\begin{frame}
    1) Разобраться в работе протокола PEAP и его применении в WPA-EAP

    2) Реализовать тестовую конфигурацию сервер-клиент
      на базе hostapd и wpa\_supplicant

    3) Настроить уязвимую конфигурацию и получить подтверждение этого
      в виде логов

    4) Реализовать MITM сервер, который компрометирует TLS в составе PEAP,
      и пробрасывает трафик протокола MSCHAPv2 между сторонами
\end{frame}

\section{Что удалось}
\begin{frame}
  1) Расширить знания про TLS, SSL, PKI

  2) Получить тестовую реализацию сервер-клиент в виде одного
    C приложения
\end{frame}

\end{document}
